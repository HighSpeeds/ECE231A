\include{"../preamble.tex"}
\title{ECE 231A HW 2}
\begin{document}
\maketitle
\section*{Problem 1}
\subsection*{(a)}
No this is not necessqarily true.  Consider the following example:

\subsection*{(b)}
We have that
\begin{align*}
    I(X_1;X_2;X_3)=&I(X_1;X_2)-I(X_1;X_2|X_3)\\
    =&H(X_1)+H(X_2)-H(X_1,X_2)-\left(
        H(X_1|X_3)-H(X_1|X_2,X_3)
    \right)\\
    =&H(X_1)+H(X_2)+H(X_3)-H(X_1,X_2)-H(X_1,X_3)\\&-H(X_2,X_3)+H(X_1,X_2,X_3)\\
\end{align*}
Since that
\begin{align*}
    I(X_1;X_2|X_3)&=H(X_1|X_3)-H(X_1|X_2,X_3)\\
    &=H(X_1,X_3)-H(X_3)-H(X_1,X_2,X_3)+H(X_2,X_3)
\end{align*}
\begin{align*}
    I(X_2;X_3|X_1)&=H(X_2|X_1)-H(X_2|X_3,X_1)\\
    &=H(X_2,X_1)-H(X_1)-H(X_1,X_2,X_3)+H(X_3,X_1)
\end{align*}
\begin{align*}
    I(X_1;X_3|X_2)&=H(X_1|X_2)-H(X_1|X_3,X_2)\\
    &=H(X_2,X_1)-H(X_2)-H(X_1,X_2,X_3)+H(X_3,X_2)
\end{align*}
Therefore we have that
$$I(X_1;X_2;X_3)=I(X_1;X_2)-I(X_1;X_2|X_3)$$
$$I(X_1;X_2;X_3)=I(X_2;X_3)-I(X_2;X_3|X_1)$$
$$I(X_1;X_2;X_3)=I(X_1;X_3)-I(X_1;X_3|X_2)$$
Since $I(X_1;X_2)\geq0$, $I(X_2;X_3)\geq0$, and $I(X_1;X_3)\geq0$, we have that
$$I(X_1;X_2;X_3)\geq-I(X_1;X_2|X_3)$$
$$I(X_1;X_2;X_3)\geq-I(X_2;X_3|X_1)$$
$$I(X_1;X_2;X_3)\geq-I(X_1;X_3|X_2)$$
Therefore we have that
$$I(X_1;X_2;X_3)\geq-\min(I(X_1;X_2|X_3),I(X_2;X_3|X_1),I(X_1;X_3|X_2))$$
\subsection*{(c)}
Once again from 
$$I(X_1;X_2;X_3)=I(X_1;X_2)-I(X_1;X_2|X_3)$$
$$I(X_1;X_2;X_3)=I(X_2;X_3)-I(X_2;X_3|X_1)$$
$$I(X_1;X_2;X_3)=I(X_1;X_3)-I(X_1;X_3|X_2)$$
Since $I(X_1;X_2|X_3)\geq0$, $I(X_2;X_3|X_1)\geq0$, and $I(X_1;X_3|X_2)\geq0$, we have that
$$I(X_1;X_2;X_3)\leq I(X_1;X_2)$$
$$I(X_1;X_2;X_3)\leq I(X_2;X_3)$$
$$I(X_1;X_2;X_3)\leq I(X_1;X_3)$$
Therefore we have that
$$I(X_1;X_2;X_3)\leq\min(I(X_1;X_2),I(X_2;X_3),I(X_1;X_3))$$
\section*{Problem 2}
We have that 
$$I(X;Y|U)=H(X|U)-H(X|Y,U)$$
since $X$ and $U$ are indepndent we have
$$I(X;Y|U)=H(X)-H(X|Y,U)$$
Since $I(X;Y,U)$ we get:
$$I(X;Y,U)=I(X;Y|U)$$

\section*{Problem 3}
\subsection*{(a)}
Let the increase in Alice's score after the ith round be 
represented by the random variable $Z_A^i$ we have that
\begin{align*}
    Z_A^i&=\begin{cases}
        4 & \text{w.p. } 1/15\\
        7 & \text{w.p. } 5/15\\
        0 & \text{w.p. } 9/15
    \end{cases}
\end{align*}
And the increase in Bob's score after the ith round be
represented by the random variable $Z_B^i$ we have that
\begin{align*}
    Z_B^i&=\begin{cases}
        3 & \text{w.p. } 1/15\\
        5 & \text{w.p. } 2/15\\
        6 & \text{w.p. } 6/15\\
        0 & \text{w.p. } 6/15
    \end{cases}
\end{align*}
Then we have that 
$$S_A^n=\sum_{i=1}^n Z_A^i$$
and
$$S_B^n=\sum_{i=1}^n Z_B^i$$
As $n\to\infty$ we have that
$$\lim_{n\to\infty}S_A^n=\lim_{n\to\infty}\sum_{i=1}^n Z_A^i=E[Z_A^i]=\boxed{2.6}$$
and
$$\lim_{n\to\infty}S_B^n=\lim_{n\to\infty}\sum_{i=1}^n Z_B^i=E[Z_B^i]=\boxed{3.2666}$$
\subsection*{(b)}
We will need to make $\alpha>7$ since we need to increase the expected value,
therefore the probabilites for $Z_A$ would not change, however instead 
of being 7 with the probability of $5/15$ we would have $\alpha$ with the
probability of $5/15$. Likewise $Z_B$ would not change. Therefore we would have that
our new 
$$E[Z_A^i]=\frac{4}{15}+\frac{5}{15}\alpha$$
To make this greater than $3.2666$ we would have that
$\boxed{\alpha>9}$, 
\section*{Problem 4}
\section*{Probelem 5}
\subsection*{(a)}
In order for $\psi=[\psi_1,\psi_2]^T$ to be a stationary distribution for 
a Markov chain we must have $\psi\Pi=\psi$, therefore we have the 
following two series of equations
$$\frac{1}{4}\psi_1+\frac{3}{4}\psi_2=\psi_1$$
$$\psi_1+\psi_2=1$$
Solving these we get $\psi_1=\psi_2=\frac{1}{2}$. Therefore the 
entropy of this $H(X)=\log_2(2)=\boxed{1\text{ shannons}}$
\end{document}
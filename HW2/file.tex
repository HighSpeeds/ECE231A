\documentclass[12pt]{article}
\author{Lawrence Liu}
\usepackage{subcaption}
\usepackage{graphicx}
\usepackage{amsmath}
\usepackage{pdfpages}
\newcommand{\Laplace}{\mathscr{L}}
\setlength{\parskip}{\baselineskip}%
\setlength{\parindent}{0pt}%
\usepackage{xcolor}
\usepackage{listings}
\definecolor{backcolour}{rgb}{0.95,0.95,0.92}
\usepackage{amssymb}
\lstdefinestyle{mystyle}{
    backgroundcolor=\color{backcolour}}
\lstset{style=mystyle}

\title{ECE 231A HW 2}
\begin{document}
\maketitle
\section*{Problem 1}
\subsection*{(a)}
No this is not necessarily true.  Consider the following example:
$X_1$ is mutually independent fo $X_2$, but conditional on $X_3$ they are
not independent.
Therefore $I(X_1;X_2)=0$ but $I(X_1;X_2|X_3)>0$. Therefore 
$I(X_1;X_2;X_3)<0$ in this case. 

\subsection*{(b)}
We have that
\begin{align*}
    I(X_1;X_2;X_3)=&I(X_1;X_2)-I(X_1;X_2|X_3)\\
    =&H(X_1)+H(X_2)-H(X_1,X_2)-\left(
        H(X_1|X_3)-H(X_1|X_2,X_3)
    \right)\\
    =&H(X_1)+H(X_2)+H(X_3)-H(X_1,X_2)-H(X_1,X_3)\\&-H(X_2,X_3)+H(X_1,X_2,X_3)\\
\end{align*}
Since that
\begin{align*}
    I(X_1;X_2|X_3)&=H(X_1|X_3)-H(X_1|X_2,X_3)\\
    &=H(X_1,X_3)-H(X_3)-H(X_1,X_2,X_3)+H(X_2,X_3)
\end{align*}
\begin{align*}
    I(X_2;X_3|X_1)&=H(X_2|X_1)-H(X_2|X_3,X_1)\\
    &=H(X_2,X_1)-H(X_1)-H(X_1,X_2,X_3)+H(X_3,X_1)
\end{align*}
\begin{align*}
    I(X_1;X_3|X_2)&=H(X_1|X_2)-H(X_1|X_3,X_2)\\
    &=H(X_2,X_1)-H(X_2)-H(X_1,X_2,X_3)+H(X_3,X_2)
\end{align*}
Therefore we have that
$$I(X_1;X_2;X_3)=I(X_1;X_2)-I(X_1;X_2|X_3)$$
$$I(X_1;X_2;X_3)=I(X_2;X_3)-I(X_2;X_3|X_1)$$
$$I(X_1;X_2;X_3)=I(X_1;X_3)-I(X_1;X_3|X_2)$$
Since $I(X_1;X_2)\geq0$, $I(X_2;X_3)\geq0$, and $I(X_1;X_3)\geq0$, we have that
$$I(X_1;X_2;X_3)\geq-I(X_1;X_2|X_3)$$
$$I(X_1;X_2;X_3)\geq-I(X_2;X_3|X_1)$$
$$I(X_1;X_2;X_3)\geq-I(X_1;X_3|X_2)$$
Therefore we have that
$$I(X_1;X_2;X_3)\geq-\min(I(X_1;X_2|X_3),I(X_2;X_3|X_1),I(X_1;X_3|X_2))$$
\subsection*{(c)}
Once again from 
$$I(X_1;X_2;X_3)=I(X_1;X_2)-I(X_1;X_2|X_3)$$
$$I(X_1;X_2;X_3)=I(X_2;X_3)-I(X_2;X_3|X_1)$$
$$I(X_1;X_2;X_3)=I(X_1;X_3)-I(X_1;X_3|X_2)$$
Since $I(X_1;X_2|X_3)\geq0$, $I(X_2;X_3|X_1)\geq0$, and $I(X_1;X_3|X_2)\geq0$, we have that
$$I(X_1;X_2;X_3)\leq I(X_1;X_2)$$
$$I(X_1;X_2;X_3)\leq I(X_2;X_3)$$
$$I(X_1;X_2;X_3)\leq I(X_1;X_3)$$
Therefore we have that
$$I(X_1;X_2;X_3)\leq\min(I(X_1;X_2),I(X_2;X_3),I(X_1;X_3))$$
\section*{Problem 2}
We have that 
$$I(X;Y|U)=H(X|U)-H(X|Y,U)$$
since $X$ and $U$ are indepndent we have
$$I(X;Y|U)=H(X)-H(X|Y,U)$$
Since $I(X;Y,U)$ we get:
$$I(X;Y,U)=I(X;Y|U)$$
Since $H(X|Y,Z)=0$, $H(X|U,Y,Z)=0$ since conditioning reduces probability
but entropy cannot be negative. Then we have
\begin{align*}
    H(X|Y,U)&=H(X,Z|Y,U)-H(Z|X,Y,U)\\
    &\leq H(X,Z|Y,U)\\
    &= H(X,Z,Y,U)-H(Y,U)\\
    &=H(X|U,Y,Z)+H(U,Y,Z)-H(Y,U)\\
    &=H(U,Y,Z)-H(Y,U)\\
    &=H(Z|U,Y)\\
    &\leq H(Z|U)
\end{align*}
Therefore we have that 
$$H(X|Y,U)\leq H(Z|U)$$
Therefore 
$$H(X)-H(X|Y,U)\geq H(X)-H(Z|U)$$
Therefore we have that 
$$I(X;Y,U)=I(X;Y|U)\geq H(X)-H(Z|U)$$
\section*{Problem 3}
\subsection*{(a)}
Let the increase in Alice's score after the ith round be 
represented by the random variable $Z_A^i$ we have that
\begin{align*}
    Z_A^i&=\begin{cases}
        4 & \text{w.p. } 1/15\\
        7 & \text{w.p. } 5/15\\
        0 & \text{w.p. } 9/15
    \end{cases}
\end{align*}
And the increase in Bob's score after the ith round be
represented by the random variable $Z_B^i$ we have that
\begin{align*}
    Z_B^i&=\begin{cases}
        3 & \text{w.p. } 1/15\\
        5 & \text{w.p. } 2/15\\
        6 & \text{w.p. } 6/15\\
        0 & \text{w.p. } 6/15
    \end{cases}
\end{align*}
Then we have that 
$$S_A^n=\sum_{i=1}^n Z_A^i$$
and
$$S_B^n=\sum_{i=1}^n Z_B^i$$
As $n\to\infty$ we have that
$$\lim_{n\to\infty}S_A^n=\lim_{n\to\infty}\sum_{i=1}^n Z_A^i=E[Z_A^i]=\boxed{2.6}$$
and
$$\lim_{n\to\infty}S_B^n=\lim_{n\to\infty}\sum_{i=1}^n Z_B^i=E[Z_B^i]=\boxed{3.2666}$$
\subsection*{(b)}
We will need to make $\alpha>7$ since we need to increase the expected value,
therefore the probabilites for $Z_A$ would not change, however instead 
of being 7 with the probability of $5/15$ we would have $\alpha$ with the
probability of $5/15$. Likewise $Z_B$ would not change. Therefore we would have that
our new 
$$E[Z_A^i]=\frac{4}{15}+\frac{5}{15}\alpha$$
To make this greater than $3.2666$ we would have that
$\boxed{\alpha>9}$, 
\section*{Problem 4}
\subsection*{(a)}
We have that $H(\hat{p}(Y_m)|Y_m)=0$ and $H(\hat{p}(Y_m)|Y_m,p)=0$, therefore
we have that $I(\hat{p}(Y_m);p|Y_m)=0$ 
\begin{align*}
    I(p;Y_m|\hat{p})+I(\hat{p}(Y_m);p)&=I(\hat{p}(Y_m);p,Y_m)
            =I(p;\hat{p}(Y_m)|Y_m)+I(p;Y_m)\\
            I(\hat{p}(Y_m);p)+I(\hat{p}(Y_m);p)&=I(p;Y_m)\\
            I(\hat{p}(Y_m);p)&\leq I(p;Y_m)
\end{align*}
With the last line coming from the fact that $I(\hat{p}(Y_m);p)\geq0$
\subsection*{(b)}
If $\sum_{i=1}^{n}X_{i}^m\neq k$ then  $\hat{p}(Y_m)\neq \frac{k}{n}$
therefore we must have 
$P[Y_m|\hat{p}(Y_m)=\frac{k}{n}]=0$ for all $\sum_{i=1}^{n}X_{i}^m\neq k$.
If $\sum_{i=1}^{n}X_{i}^m=k$ then we have $n \choose k$ possible 
$Y_m$, since $X_{i}^m$ is iid, each of these are equally probable therefor we have
$$P[Y_m|\hat{p}(Y_m)=\frac{k}{n}]=\begin{cases}
    \frac{1}{{n \choose k}} & \text{if } \sum_{i=1}^{n}X_{i}^m=k\\
    0 & \text{otherwise}
\end{cases}$$
\subsection*{(c)}
Let $\hat{p}=\hat{p}(Y_m)$ First let us prove that 
$P(Y_m|\hat{p},p)=P(Y_m|\hat{p})$.
\begin{align*}
    P(Y_m|\hat{p},p)&=P(Y_m|\hat{p})\\
    P(Y_m,\hat{p},p)&=P(Y_m|\hat{p})p(\hat{p}|p)p(p)
\end{align*}
For all the cases of $Y_m$ such that $\sum_{i=1}^{n}X_{i}^m\neq k$ we have that
$P(Y_m,\hat{p},p)=0$ and $P(Y_m|\hat{p})=0$, therefore we will only need to Consider
the cases where $\sum_{i=1}^{n}X_{i}^m=k$, for these cases we have
\begin{align*}
    P(Y_m,\hat{p},p)&=\frac{1}{{n\choose k}}p(\hat{p}|p)p(p)
\end{align*}
Since $X_{i}^m$ is iid we have that $p(\hat{p}|p)={n\choose k}p(Y_m|p)$, therefore
Therefore we have that 
\begin{align*}
    P(Y_m,\hat{p},p)&=p(Y_m|p)p(p)\\
    &=P(Y_m,p)
\end{align*}
This is true since $\hat{p}$ is a function of $Y_m$. Therefore we have that,
$p$, $\hat{p}$ and $Y_m$ form the following Markov chain:
$$p\rightarrow \hat{p}\rightarrow Y_m$$
Therefore from the data processing inequality we have 
$$I(\hat{p}(Y_m);p)\geq I(p;Y_m)$$
therefore from this and problem 4(a) we have that
$$I(\hat{p}(Y_m);p|Y_m)= I(p;Y_m)$$
\subsection*{(d)}
From part (c) we have that $I(p;\hat{p}(Y_i))=I(p;Y_i)$, therefore we have that
we just need to prove 
$$I(Y_{i};p)\geq I(Y_{j};p)$$
If we prove this inequality for $j=i+1$, we have
proved it for all $j>i$. Since $H(Y_{i+1}|Y_i,p)=H(Y_{i+1}|Y_i)$, therefore we have that 
$I(Y_{i+1};p|Y_i)=0$ therefore we have
\begin{align*}
    I(Y_i;p|Y_{i+1})+I(Y_{i+1};p)&=I(Y_i;p,Y_{i+1})=I(Y_{i+1};p|Y_i)+I(Y_{i};p)\\
    I(Y_i;p|Y_{i+1})+I(Y_{i+1};p)&=I(Y_{i};p)\\
    I(Y_{i+1};p)&\leq I(Y_{i};p)
\end{align*}
With the last line coming from the fact that $I(Y_i;p|Y_{i+1})\geq0$
Therefore we have that 
$$I(Y_{i};p)\geq I(Y_{j};p)$$
for all $j>i$.
\subsection*{(e)}
From part (d) we have that the $Y_i$'s form a markov chain
thus we have that 
$$Y_i\rightarrow Y_{i+1}\rightarrow Y_{i+2}\rightarrow \cdots Y_{m-1}\rightarrow Y_m$$
Since $\hat{p}(Y_m)$ is a function of $Y_m$ we have that 
$$Y_i\rightarrow Y_{i+1}\rightarrow \cdots Y_{m-1}\rightarrow Y_m\rightarrow \hat{p}(Y_m)$$
Forms a markov chain as well. We have
\begin{align*}
    I(\hat{p}(Y_m);Y_{m-1}|Y_i)&\geq I(\hat{p}(Y_m);Y_{m-1}|Y_j)\\
    H(\hat{p}(Y_m)|Y_i)-H(\hat{p}(Y_m)|Y_i,Y_{m-1})&\geq H(\hat{p}(Y_m)|Y_j)-H(\hat{p}(Y_m)|Y_j,Y_{m-1})\\
    H(\hat{p}(Y_m)|Y_i)-H(\hat{p}(Y_m)|Y_{m-1})&\geq H(\hat{p}(Y_m)|Y_j)-H(\hat{p}(Y_m)|Y_{m-1})\\
    H(\hat{p}(Y_m)|Y_i)&\geq H(\hat{p}(Y_m)|Y_j)
\end{align*}
We can see that this is intuitively true, since $Y_{1}$ is "further" from 
$Y_{m}$ and $\hat{p}(Y_m)$ than $Y_{j}$ is, therefore it has less 
"determination" on $Y_{m}$ and $\hat{p}(Y_m)$.
\section*{Problem 5}
\subsection*{(a)}
In order for $\mu=[\mu_1,\mu_2]^T$ to be a stationary distribution for 
a Markov chain we must have $\mu^T\Pi=\mu^T$, therefore we have the 
following two series of equations
$$\frac{1}{4}\mu_1+\frac{2}{3}\mu_2=\mu_1$$
$$\frac{3}{4}\mu_1+\frac{1}{3}\mu_2=\mu_2$$
Solving these we get, and applying the condition that $\mu_1+\mu_2=1$,
we get $\mu_1=\frac{\frac{2}{3}}{\frac{3}{4}+\frac{2}{3}}=\boxed{0.470}$ and
$\mu_2=\frac{\frac{3}{4}}{\frac{3}{4}+\frac{2}{3}}=\boxed{0.529}$ Therefore the 
entropy of this $H(X)=-0.470\log_2(0.470)-0.529\log_2(0.529)=\boxed{0.9975\text{ shannons}}$
\subsection*{(b)}
We have that 
\begin{align*}
    \mathcal{H}(x)&=-\sum_{i}\mu_i\sum_{j}P_{ij}\log_2(P_{ij})\\
    &=-0.470\left(\frac{1}{4}\log_2\left(\frac{1}{4}\right)+
    \frac{3}{4}\log_2\left(\frac{3}{4}\right)\right)-0.529\left(\frac{2}{3}\log_2\left(\frac{2}{3}\right)+
    \frac{1}{3}\log_2\left(\frac{1}{3}\right)\right)\\
    &=\boxed{0.8679345589507923}
\end{align*}
This is less than $H(X)$ since 
$\mathcal{H}(x)=\lim_{n\to\infty}H(X_n|X_{n-1},\dots,X_1)$, 
since conditioning reduces entropy, thus we have that 
$H(X_n|X_{n-1},\dots,X_1)<H(X)$.
\subsection*{(c)}
Since we can determine $X_n$ given $Y_n,...,Y_1$ and since 
$Y_n$ is a deterministic function we have
$$H(Y_n|Y_{n-1},\dots,Y_1)=H(X_{n}|X_{n-1},...,X_1)$$
And therefore we have that 
$\mathcal{H}(Y)=\mathcal{H}(X)=\boxed{0.8679345589507923}$
\subsection*{(d)}
We have that 
$$H(Y_n,X_n,X_{n-1})=H(Y_n|X_n,X_{n-1})+H(X_n,X_{n-1})=H(Y_n)+H(X_n,X_{n-1}|Y_n)$$
Therefore we get 
$$H(X_n,X_{n-1})=H(Y_n)+H(X_n,X_{n-1}|Y_n)$$
Since XOR if given one of the values, we can imediately determine the other
value we have 
\begin{align*}
    H(X_n,X_{n-1})&=H(Y_n)+H(X_{n-1})\\
    H(X_n|X_{n-1})&=H(Y_n)\\
    H(Y_n)&=\mathcal{H}(X)\\
    H(Y_n)&=\boxed{0.8679345589507923}
\end{align*}



\end{document}
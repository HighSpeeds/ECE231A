\documentclass[12pt]{article}
\author{Lawrence Liu}
\usepackage{subcaption}
\usepackage{graphicx}
\usepackage{amsmath}
\usepackage{pdfpages}
\newcommand{\Laplace}{\mathscr{L}}
\setlength{\parskip}{\baselineskip}%
\setlength{\parindent}{0pt}%
\usepackage{xcolor}
\usepackage{listings}
\definecolor{backcolour}{rgb}{0.95,0.95,0.92}
\usepackage{amssymb}
\lstdefinestyle{mystyle}{
    backgroundcolor=\color{backcolour}}
\lstset{style=mystyle}

\title{ECE 231A HW 5}
\begin{document}
\maketitle
\section*{Problem 1}
\subsection*{(a)}
First let us prove that 
$$\min_{i}\frac{a_i}{b_i}\leq\frac{\sum_{i=1}^{n}a_i}{\sum_{i=1}^{n}b_i}\leq\max_{i}\frac{a_i}{b_i}$$
We can prove this through induction, we already have that 
the base case when $n=2$ is true. Now
we consider the case the case of $n+1$, let us 
arrange the $a_i$ and $b_i$ in such a way such that 
$\frac{a_{n+1}}{b_{n+1}}$ be the minimum of the $n+1$ $\frac{a_{i}}{b_{i}}$'s. 
Then we have that  
$$\frac{a_{n+1}}{b_{n+1}} \leq \min_{1\leq i\leq n}\frac{a_i}{b_i}
\leq \frac{\sum_{i=1}^{n}a_i}{\sum_{i=1}^{n}b_i}$$
Therefore we have that 
$$\min_{i}\frac{a_i}{b_i}\leq\frac{\sum_{i=1}^{n+1}a_i}{\sum_{i=1}^{n+1}b_i}
\leq\frac{\sum_{i=1}^{n}a_i}{\sum_{i=1}^{n}b_i}\leq \max_{i}\frac{a_i}{b_i}$$
Thus we have proven that $\frac{\sum_{i=1}^{n}a_i}{\sum_{i=1}^{n}b_i}$ 
is bounded by the minimum and maximum of the $\frac{a_i}{b_i}$'s. Using this property we get
that 
$$\frac{\sum_{x\in \mathcal{X}}\tilde{P}_X(x)D(W_{Y|X}||P_Y)}
{\sum_{x\in \mathcal{X}}\tilde{P}_X(x)c(x)}\leq 
\max_{x\in \mathcal{X}} \frac{\tilde{P}_X(x)D(W_{Y|X}||P_Y)}{\tilde{P}_X(x)c(x)}=\max_{x\in \mathcal{X}} \frac{D(W_{Y|X}||P_Y)}{c(x)}$$
\subsection*{(b)}
We have that 
\begin{align*}
    \sum_{x\in \mathcal{X}}\tilde{P}_X(x)D(W_{Y|X}||P_Y)-\sum_{x\in \mathcal{X}}\tilde{P}_X(x)D(W_{Y|X}||\tilde{P}_Y) &= \sum_{x\in \mathcal{X}}\tilde{P}_X(x)\left(D(W_{Y|X}||P_Y)-D(W_{Y|X}||P_Y)\right)\\
    &= \sum_{x\in \mathcal{X}}\tilde{P}_X(x)\sum_{y\in \mathcal{Y}}W_{Y|X}(y|x)\log\left(\frac{\tilde{P}_Y(y)}{P_Y(y)}\right)\\
    &=\sum_{y\in \mathcal{Y}}\log\left(\frac{\tilde{P}_Y(y)}{P_Y(y)}\right)\sum_{x\in \mathcal{X}}\tilde{P}_X(x)W_{Y|X}(y|x)\\
    &=\sum_{y\in \mathcal{Y}}\tilde{P}_Y(y)\log\left(\frac{\tilde{P}_Y(y)}{P_Y(y)}\right)\\
    &=D_{KL}(\tilde{P}_Y||P_Y)\geq 0
\end{align*}
Therefore we have equality only happens when $\tilde{P}_Y=P_Y$.
From this we get that
$$\sum_{x\in \mathcal{X}}\tilde{P}_X(x)D(W_{Y|X}||P_Y)\geq \sum_{x\in \mathcal{X}}\tilde{P}_X(x)D(W_{Y|X}||\tilde{P}_Y)$$
Thus we have that
$$
\frac{\sum_{x\in \mathcal{X}}\tilde{P}_X(x)D(W_{Y|X}||\tilde{P}_Y)}{\sum_{x\in \mathcal{X}}\tilde{P}_X(x)c(x)}\leq \frac{\sum_{x\in \mathcal{X}}\tilde{P}_X(x)D(W_{Y|X}||P_Y)}{\sum_{x\in \mathcal{X}}\tilde{P}_X(x)c(x)}\leq \max_{x\in \mathcal{X}} \frac{D(W_{Y|X}||\tilde{P}_Y)}{c(x)}$$
\subsection*{(c)}
We have that
$$\sum_{x\in \mathcal{X}}\tilde{P}_X(x)D(W_{Y|X}||P_Y)\leq \sum_{x\in \mathcal{X}}\tilde{P}_X(x)c(x)\lambda$$
Thus we have that
$$\frac{\sum_{x\in \mathcal{X}}\tilde{P}_X(x)D(W_{Y|X}||P_Y)}{\sum_{x\in \mathcal{X}}\tilde{P}_X(x)c(x)}\leq \lambda$$
With equality if and only if $\tilde{P}_Y=P_Y$ and $\tilde{P}_X(x)=0$ for all 
$x$ where $P^*_X(x)=0$, because then we will have that 
$$\sum_{x\in \mathcal{X}}\tilde{P}_X(x)D(W_{Y|X}||P_Y)= \sum_{x\in \mathcal{X}}\tilde{P}_X(x)c(x)\lambda$$
\subsection*{(d)}
We have that 
$$I(X;Y)=\sum_{x\in \mathcal{X}}\tilde{P}_X(x)D(W_{Y|X}||P_Y)$$
and 
$$E[c(x)]=\sum_{x\in \mathcal{X}}\tilde{P}_X(x)c(x)$$
Thus we have that from part (c)
$$C_{cost}=\max_{P_X(x)}\frac{I(X;Y)}{E[c(x)]}=\lambda$$
\section*{Problem 2}
\subsection*{(a)}
We have that in order to maximize the 
differential entropy of $Z$ we want to maximize
the variance of $Z$, this is given by 
$$\text{Var}(Z)=E[(Z_1+Z_2)^2]-(E[Z_1+Z_2])^2$$
Since $Z_1$ is independent of $Z_2$ we have that
$E[Z_1Z_2]=E[Z_1]E[Z_2]$, thus we get that 
$$\text{Var}(Z)=E[Z_1^2]+E[Z_2^2]-E^2[Z_1]-E^2[Z_2]\leq 2\sigma^2 -E^2[Z_1]-E^2[Z_2]$$
Therefore in order for this to be maximized, we want $E[Z_1]=E[Z_2]=0$. 
Thus we get that both $Z_1$ and $Z_2$ are Gaussian with mean 0. And that 
the maximal differential entropy of $Z$ is given by $\frac{1}{2}\log(4\pi e\sigma^2)$.
\subsection*{(b)}
Once again we want to maximize the differential entropy of $Z$,
and thus we want to maximize the variance of $Z$, this is given by
$$\text{Var}(Z)=E[(\sum_{i=1}^{n}Z_i)^2]-(E[\sum_{i=1}^{n}Z_i])^2$$
We have that 
$$\left(\sum_{i=1}^{n}Z_i\right)^2=\sum_{i=1}^{n}Z_i^2+\sum_{i\neq j}Z_iZ_j$$
And
$$E^2[\sum_{i=1}^{n}Z_i]=\sum_{i=1}^{n}E^2[Z_i]+\sum_{i\neq j}E[Z_i]E[Z_j]$$
Since $Z_i$'s are independent we have that
$$E[Z_iZ_j]-E[Z_i]E[Z_j]=0$$
For all $i\neq j$, thus we get that
$$\text{Var}(Z)=\sum_{i=1}^{n}E[Z_i^2]-\sum_{i=1}^{n}E^2[Z_i]=n\sigma^2-\sum_{i=1}^{n}E^2[Z_i]$$
Therefore we must have that $E[Z_i]=0$ for all $i$, thus we get that
the maximum differential entropy of $Z$ is given by $\boxed{\frac{1}{2}\log(2\pi e n\sigma^2)}$.
\subsection*{(c)}
We have that 
$$\text{Var}(Z)=E[(\sum_{i=1}^{n}Z_i)^2]-(E[\sum_{i=1}^{n}Z_i])^2$$
We can rewrite this as 
\begin{align*}
    \text{Var}(Z)&=E[\sum_{i=1}^{n}Z_i^2]-E[\sum_{i=1}^{n}Z_i]^2+\sum_{i\neq j}E[Z_iZ_j]-E[Z_i]E[Z_j]\\
    &=\sum_{i=1}^{n}Var(Z_i)+\sum_{i\neq j}Cov(Z_i,Z_j)
\end{align*}
We have that $Cov(Z_i,Z_j)\leq \sqrt{Var(Z_i)Var(Z_j)}$, and since 
$Var(Z_i)\leq \sigma^2$ we get that
$$Var(z)\leq n^2\sigma^2$$
With equality only happening when $E[Z_i]=0$, therefore we get that 
the joint distribution of $Z_i$'s as expressed by a vector  
$\mathbf{Z}=(Z_1,Z_2,\ldots,Z_n)$ is distrubuted as a multivariate Gaussian with mean $\mu=[0,\ldots,0]$ and covariance matrix
$\Sigma$ where $\Sigma_{ij}=\sigma^2$ for all $1\leq i,j\leq n$. And with this the 
maximum differential entropy of $Z$ is given by $\boxed{\frac{1}{2}\log_2(2\pi e n^2\sigma^2)}$.
\section*{Problem 3}
\subsection*{(a)}
We have:
\begin{align*}
    I(Y_1,Y_2;X)&=H(Y_1,Y_2)-H(Y_1,Y_2|X)\\
    &=H(Y_1,Y_2)-H(Z_1,Z_2)\\
    &=H(Y_1,Y_2)-H(Z_1)-H(Z_2)\\
    &=H(Y_1,Y_2)-\log(2\pi e\sigma^2)
\end{align*}
We have that with $\textbf{Y}=[Y_1,Y_2]$, and since 
$E[Y_1^2]=E[Y_2^2]=\sigma^2+P$ and $E[Y_1Y_2]=P$ we get that
$K=\begin{bmatrix}
    \sigma^2+P & P\\
    P & \sigma^2+P
\end{bmatrix}$ therefore we get
that 
$$H(Y_1,Y_2)=\frac{1}{2}\log_2(2\pi e\sigma^2 det(K))=\frac{1}{2}\log_2((2\pi e)^2\sigma^2(\sigma^2+2P))$$
Thus we get that 
\begin{align*}
    I(Y_1,Y_2;X)&\leq \frac{1}{2}\log\left(1+\frac{2P}{\sigma^2}\right)
\end{align*}
Therefore we get that the channel capacity is $\boxed{\frac{1}{2}\log\left(1+\frac{2P}{\sigma^2}\right)}$
and that we can achieve this capacity by having an X distributed as a Gaussian with mean 0 and variance $P$.
\subsection*{(b)}
We have that we want to maximize 
\begin{align*}
    I(Y_1,Y_2,Y_3,Y_4;X)&=H(Y_1,Y_2,Y_3,Y_4)-H(Y_1,Y_2,Y_3,Y_4|X)\\
    &=H(Y_1,Y_2,Y_3,Y_4)-H(Z_1,Z_2,Z_3,Z_4)\\
    &\leq H(Y_1,Y_2)+H(Y_3,Y_4)-H(Z_1)-H(Z_2)-H(Z_3)-H(Z_4)\\
    &=H(Y_1,Y_2)+H(Y_3,Y_4)-\log(2\pi e\sigma^2)-\log(2\pi e\sigma^2)\\
    &\leq \frac{1}{2}\log\left(1+\frac{2P_1}{\sigma_1^2}\right)
    +\frac{1}{2}\log\left(1+\frac{2P_2}{\sigma_2^2}\right)
\end{align*}
Therefore we have the optimization problem for the channel capacity is just
maximizing $\frac{1}{2}\log\left(1+\frac{2P_1}{\sigma_1^2}\right)
+\frac{1}{2}\log\left(\frac{2P_2}{\sigma_2^2}\right)$ subject to 
the constrain that $P_1+P_2=P$.
\subsection*{(c)}
Using a langrangian multiplier we get that to find the optimal solution we must 
find $P_1$ and $P_2$ that satisfy 
$$\nabla f(P_1,P_2)=0$$
where 
$$f(P_1,P_2)=\frac{1}{2}\log\left(\frac{2P_1}{\sigma_1^2}+1\right)+\frac{1}{2}
\log\left(\frac{2P_2}{\sigma_2^2}+1\right)+\lambda(P_1+P_2-P)$$
We have that 
$$\frac{\partial f}{\partial P_1}=\frac{1}{2P_1+\sigma_1^2}+\lambda=0$$
and
$$\frac{\partial f}{\partial P_2}=\frac{1}{2P_2+\sigma_2^2}+\lambda=0$$
Therefore we get that 
$$P_1=-\frac{1}{2\lambda}-\frac{\sigma_1^2}{2}$$
$$P_2=-\frac{1}{2\lambda}-\frac{\sigma_2^2}{2}$$

Thus we get that 
$$\lambda=-\frac{2}{\sigma^2_1+\sigma^2_2+2P}$$
and
$$P_1=\frac{\sigma^2_1+\sigma^2_2+2P}{4}-\sigma_1^2$$
$$P_2=\frac{\sigma^2_1+\sigma^2_2+2P}{4}-\sigma_2^2$$
However we need to have that $P_1\geq 0$ and $P_2\geq 0$ and therefore we have that 
$$P_1=\boxed{\left(\nu-\frac{\sigma_1^2}{2}\right)^{+}}$$
$$P_2=\boxed{\left(\nu-\frac{\sigma_2^2}{2}\right)^{+}}$$
where $\nu$ is chosen such that $(\nu-\frac{\sigma_1^2}{2})^{+}+(\nu-\frac{\sigma_2^2}{2})^{+}=P$, and 
$(x)^{+}=\max\{0,x\}$. This satisfies the Kuhn-Tucker conditions. Therefore we get that
the optimal channel capacity is:
$$\boxed{\log\left(\frac{(2\nu-\sigma_1^2)^{+}}{\sigma_1^2}+1\right)+
\log\left(\frac{(2\nu-\sigma_2^2)^{+}}{\sigma_2^2}+1\right)}$$
\subsection*{(d)}
We have that if $P=3$ and $P_2=0$ then $P_1=3$ and we get that $\nu=3+1=4$ and thus we have that 
$\sigma_2^2\geq 2\nu=\boxed{8}$
\section*{Problem 4}
\subsection*{(a)}
We have that we want to minimize
$$I(X;\hat{X}|Y)=H(X|Y)-H(X|Y,\hat{X})$$
When $Y\neq E$, the entropy of $H(X|Y\neq E)=H(X|Y\neq E,\hat{X})=0$,
thus we get
$$I(X;\hat{X}|Y)=(p)(H(X|Y=E)-H(X|Y=E,\hat{X}))=(p)(H(X)-H(X|\hat{X}))
$$
From cover and thomas, we get that the minimum of $H(X)-H(X|\hat{X})=H(\frac{1}{2})-H(D_E)$.
For some $D_E$, that is expected distortion given that $Y=E$, we get that
if $D>\frac{1}{2}p$ then we can set $D_E=\frac{1}{2}$, and thus 
we will have that $H(X)-H(X|\hat{X})=H(\frac{1}{2})-H(\frac{1}{2})=0$.
If $D<\frac{1}{2}p$ then we want to "lump" all of the 
distortion to the case when $Y=E$, therefore we will have
$D_E=\frac{D}{p}$ and thus we will have that
$$R_{X|Y}(D)=\boxed{\begin{cases}
    p\left(1-H\left(\frac{D}{p}\right)\right) & D<\frac{1}{2}p\\
    0 & D\geq \frac{1}{2}p
\end{cases}}$$
\subsection*{(b)}
From the previous part we once again have 
$$I(X;\hat{X}|Y)=p(H(X)-H(X|\hat{X}))$$
Since the $d(0,1)=d(1,0)=\infty$ we have that 
if $\hat{X}\neq E$,  $\hat{X}$ must be equal to $X$, 
Therefore $H(X|\hat{X}\neq E)=0$. Therefore we get that 
$$H(X|\hat{X})\leq p_EH(X|\hat{X}=E)=DH(X)=p_EH(\frac{1}{2})=D$$
Where $p_E$ is the probability that $\hat{X}=E$ given that 
$Y=E$. We have that we want $p_E=1$, but this is only possible
if $D>p$ otherwise we will have $p_E=\frac{D}{p}$.
Thus we have that the rate distortion function is 
$$R_{X|Y}(D)=\boxed{\begin{cases}
    p(1-\frac{D}{p}) & D<p\\
    0 & D\geq p
\end{cases}}$$

\section*{Problem 5}
\subsection*{(a)}
We have that $H(X)=\frac{1}{2}\log((2\pi e)^2 2)$ and let 
$E((X_1-\underline{X_1})^2)=\sigma_1^2$ and $E((X_2-\underline{X_2})^2)=\sigma_2^2$ for 
some $\sigma_1^2+\sigma_2^2=\frac{5}{2}$. We have that
\begin{align*}
    I(X;\underline{X})&=H(X_1)-H(X|\underline{X})\\
    &=\frac{1}{2}\log((2\pi e)^2 2)-H(X-\underline{X}|\underline{X})\\
    &\leq\frac{1}{2}\log((2\pi e)^2 2)-H(X-\underline{X})\\
    &=\frac{1}{2}\log((2\pi e)^2 2)-H(X_1-\underline{X_1})-H(X_2-\underline{X_2})\\
    &=\frac{1}{2}\log((2\pi e)^2 2)-\frac{1}{2}\log((2\pi e)^2 \sigma_1^2)-\frac{1}{2}\log((2\pi e)^2 \sigma_2^2)\\
\end{align*}
Therefore we have that we want to minimize $\frac{1}{2}\log\left(\frac{4}{\sigma_1^2\sigma_2^2}\right)$
subject to the constraint that $\sigma_1^2+\sigma_2^2=\frac{5}{2}$.
\subsection*{(b)}
To solve for $\sigma_1^2$ and $\sigma_2^2$ we use a langrangian multiplier and get that we want 
to find the $\sigma_1^2$ and $\sigma_2^2$ that satisfy
$$\nabla f(\sigma_1^2,\sigma_2^2)=0$$
where
$$f(\sigma_1^2,\sigma_2^2)=\frac{1}{2}\log\left(\frac{2}{\sigma_1^2\sigma_2^2}\right)+\lambda(\sigma_1^2+\sigma_2^2-\frac{5}{2})$$
We have that
$$\frac{\partial f}{\partial \sigma_1^2}=\lambda-\frac{1}{2\sigma_1^2}=0$$
and
$$\frac{\partial f}{\partial \sigma_2^2}=\lambda-\frac{1}{2\sigma_2^2}=0$$
Therefore we get that
$$\sigma_1^2=\frac{1}{2\lambda}$$
and
$$\sigma_2^2=\frac{1}{2\lambda}$$
Thus we get that
$$\lambda=\frac{1}{\frac{5}{2}}=\frac{2}{5}$$
and thus we get that
$$\sigma_1^2=\boxed{\frac{5}{4}}$$
$$\sigma_2^2=\boxed{\frac{5}{4}}$$
And thus we will send $X_1$ through a gaussian channel with 
$Z_1\sim \mathcal{N}(0,\frac{5}{4})$ and $X_2$ through a gaussian channel with
$Z_2\sim \mathcal{N}(0,\frac{5}{4})$.
\subsection*{(c)}
We have that we can transform $Y_1$ abd $Y_2$ into 
$$Y'_1=Y_1+Y_2$$
$$Y'_2=Y_1-Y_2$$
These are independent, and both are 
gaussian with mean 0 and variance of $4$ and $2$ respectively.
So we can send $Y'_1$ and $Y'_2$ through gaussian channels with 
$Z'_1\sim \mathcal{N}(0,\sigma_1^2)$ and $Z'_2\sim \mathcal{N}(0,\sigma_2^2)$ respectively.with 
outputs of $\hat{Y}'_1$ and $\hat{Y}'_2$, 
then we will have that the rate disortion measure is:
\begin{align*}
D&=E\left[\left(\frac{Y'_1+Y'_2}{2}-\frac{\hat{Y}'_1+\hat{Y}'_2}{2}\right)^2\right]
+E\left[\left(\frac{Y'_1-Y'_2}{2}-\frac{\hat{Y}'_1-\hat{Y}'_2}{2}\right)^2\right]\\
&=\frac{\sigma_1^2}{2}+\frac{\sigma_2^2}{2}
\end{align*}

Thus we have that
in order to maximize the rate disortion function, 
we must maximize 
$$\frac{1}{2}\log((2\pi e)^2 8)-\frac{1}{2}\log((2\pi e)^2 \sigma_1^2)-\frac{1}{2}\log((2\pi e)^2 \sigma_2^2)$$
With the constraint that $\sigma_1^2+\sigma_2^2=5$. We have that
using a langrangian multiplier we get that we want to find the $\sigma_1^2$ and $\sigma_2^2$ that satisfy
$$\nabla f(\sigma_1^2,\sigma_2^2)=0$$
where
$$f(\sigma_1^2,\sigma_2^2)=\frac{1}{2}\log\left(\frac{8}{\sigma_1^2\sigma_2^2}\right)+\lambda(\sigma_1^2+\sigma_2^2-5)$$
Once again we get that 
$$\frac{\partial f}{\partial \sigma_1^2}=\lambda-\frac{1}{2\sigma_1^2}=0$$
and
$$\frac{\partial f}{\partial \sigma_2^2}=\lambda-\frac{1}{2\sigma_2^2}=0$$
Therefore we get that
$$\sigma_1^2=\frac{1}{2\lambda}$$
and
$$\sigma_2^2=\frac{1}{2\lambda}$$
Thus we get that
$$\lambda=\frac{1}{5}$$
and thus we get that
$$\sigma_1^2=\boxed{\frac{5}{2}}$$
$$\sigma_2^2=\boxed{\frac{5}{2}}$$
\end{document}